\newpage
\begingroup
	\begin{center}
	\large{МАТЕМАТИЧЕСКИЙ КРУЖОК}
	\end{center}
\endgroup

\begin{multicols}{2}
\begin{figure}[H]
\includegraphics[width = 0.5\textwidth]{image}
\end{figure}

 \vspace*{10mm}
\setcounter{page}{37}
\begin{textbf}
{Вычисление сумм - один из важнейших и интереснейших
вопросов математики. Существуют разные методы вычисления сумм.
В статье рассказывается о двух из них. 
} 
\end{textbf}

1. В математике и ее многочисленных приложениях для сокращенной записи суммы употребляется специальный знак. 
Это $\Sigma$ - буква греческого алфавита  <<сигма>>. Запись суммы посредством знака $\Sigma$ часто бывает очень удобной.
Познакомимся с этим знаком. 

Пусть дана сумма вида \\
$$a_1 + a_2 + ... + a_n.$$
Все слагаемые этой суммы обозначены одной буквой, для отличия использованы индексы.
Данную сумму сокращенно можно записать в следующем виде:
$$\sum_{k=1}^{n} a_k.$$
Читается: <<сигма $a_k$, \textit{k} меняется от 1 до \textit{n}>>. Для такой записи берется <<типичное>> слагаемое
суммы, в нашем случае $a_k$ *), перед ним пишется знак $\Sigma$ и указываются границы измерения \textit{k}. Например, запись
$$\sum_{k=1}^{10} k.$$
означает сумму 1 + 2 + 3 + 4 + 5 + 6 + 7 + 8 \underline{+ 9 + 10,} \\
\small{*) $a_k$ называется общим членом суммы}\\
\begin{flushright}
\textbf{А. Д. Бендукиндзе, \\
А. К. Сулаквелидзе}
\end{flushright}
 \vspace*{55mm}
а запись
$$\sum_{k=1}^{n} \frac{1}{k(k+1)}$$
- сумму
$$\frac{1}{1*2} + \frac{1}{2*3} + ... + \frac{1}{(n-1)*n} + \frac{1}{n*(n+1)}$$

Нетрудно проверить следующие свойства знака $\Sigma$:
$$\sum_{k=1}^{n} 1 = n, \sum_{k=1}^{n} (ca_k) = c \sum_{k=1}^{n} a_k,$$
$$\sum_{k=1}^{n} (a_k + b_k) = \sum_{k=1}^{n} a_k + \sum_{k=1}^{n} b_k.$$

Проверим, к примеру, второе свойство. \\
По определению
$$\sum_{k=1}^{n} ca_k = ca_1 + ca_2 + ... + ca_n,$$
поэтому, согласно известному свойству сумммы, имеем:
$$\sum_{k=1}^{n} ca_k =  c(a_1 + a_2 + ... + a_n).$$
Но выражение в скобках есть не что иное, как
$$\sum_{k=1}^{n} a_k$$
и, следовательно, 
$$\sum_{k=1}^{n} ca_k = c \sum_{k=1}^{n} a_k$$
\end{multicols}


\newpage
\setcounter{page}{5}
\begin{multicols}{2}
\setlength{\columnsep}{3cm}
\large{П р и м е р.}
\begin{tabbing}
MMMMM \= MMMMMMMMMMMM \= MMMM \kill
 \footnotesize{№} \> \footnotesize{Занумерованные дроби} \> \footnotesize{$\beta$=0,}\\
MMMM \= MMMMMMMMMMMMMM \= M \kill
 1. \> 0, \fbox{1} 2 3 4 5 6 7 8 0 1... \>    2\\
 2. \> 0, 2 \fbox{4} 6 1 4 5 3 2 1 4... \>    5\\
 3. \> 0, 1 3 \fbox{2} 4 5 3 1 7 8 2... \>    3\\
 4. \> 0, 0 1 4 \fbox{6} 7 2 7 8 0 1... \>    7\\
 5. \> 0, 4 2 3 1 \fbox{2} 1 4 5 6 0... \>    3\\
 6. \> 0, 5 6 7 2 4 \fbox{5} 8 0 1 2... \>    6\\
 7. \> 0, 2 4 3 5 6 7 \fbox{8} 0 2 4... \>    0\\
 8. \> 0, 7 5 4 6 2 1 2 \fbox{0} 1 2... \>    1\\
 9. \> 0, 8 0 8 0 1 0 2 2 \fbox{2} 2... \>    3\\
. . . . . . . . . . . . . . . . . . . . . . . . \= MM \=   . . .  \kill
. . . . . . . . . . . . . . . . . . . . . . . .  \>        \>   . . .  
\end{tabbing}
$$\beta=0,25376013...$$

З а д а ч а  1. Доказать, что построенная дробь $\beta$ не входит в ихсодную последовательность. 

Итак, второй факт доказан, перейдем к первому. 

Сначала обсудим поучительную историю об универсальной библиотеке. 
сколь угодно длинные пробелы (в частности, благодаря этому в число пятисотстраничных книг можно включить книги, состоящие из меньшего числа непустых страниц). В результате книгу можно предстваить себе как последовательность из 50 Х 40 Х 500 = $10^{6}$ знаков, каждый из которых может быть одним из 100 знаков наборной азбуки (литер), т. е. как одно слово из миллиона букв в языке. алфавит которого состоит из 100 букв. Обратите внимание, что это свдение стало возможным благодаря введению знака пробела, иначе последовательность знаков могла бы не определять книгу однозначно(ее можно по-разному разбить на слова). 

З а д а ч а  2. Доказать, что число различных слов длины n в языке, алфавит которого состоит из $k$ букв, равно $k^{n}$ *).
\end{multicols}








